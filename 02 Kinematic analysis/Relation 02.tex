\documentclass[10.5pt, twocolumn]{article}

\usepackage[english]{babel}
\usepackage{graphicx}
\usepackage{imakeidx}
\usepackage{mathrsfs, amsmath}
\usepackage{systeme}
\usepackage{array}
\usepackage[utf8]{inputenc}
\usepackage{siunitx}
\usepackage{booktabs}
\usepackage{adjustbox}
\usepackage{geometry}
\geometry{a4paper,total={145mm,210mm}}
\usepackage{makecell}
\usepackage{afterpage}
\usepackage{listings}
\usepackage{subcaption}
\usepackage[toc,page]{appendix}
\usepackage[table]{xcolor}
\usepackage{pifont} % ding symbols
\usepackage{tikz}
\usepackage{changepage}
\usepackage{multirow} % multi row in tables
\usepackage{booktabs}
\usepackage{textcomp} % registered and copyright symbol
\usepackage{lscape} % vertical instead to horizontal
\usepackage{longtable} % for more page tables
\usepackage{eurosym}
\usepackage{lmodern}
\usepackage{amstext}
\usepackage{pdfpages} % import pdf pages
%\usepackage[hidelinks]{hyperref} % delete ugly hyperref borders of hyperlink
\usepackage{hyperref} % internal hyperlinks
\usepackage{titling} % titles
\usepackage{titlesec} % subtitles
\usepackage{blindtext} % for casual texts
\usepackage{dblfloatfix} % forces image at bottom in two-column files
\usepackage{gensymb} % standard unit of measurement
\usepackage{enumitem}

\DeclareRobustCommand{\officialeuro}{%
  \ifmmode\expandafter\text\fi
  {\fontencoding{U}\fontfamily{eurosym}\selectfont e}}



\makeindex[columns=2, title=Indice alfabetico, options= -s mystyle.ist, intoc]

\newcommand*{\Scale}[2][4]{\scalebox{#1}{\ensuremath{#2}}}
\renewcommand*\contentsname{Indice}
\newcommand*\NewPage{\newpage\null\thispagestyle{empty}\newpage}
\newcommand{\Virgolette}[1]{``#1''}
\newcommand*\circled[1]{\tikz[baseline=(char.base)]{
	\node[shape=circle,draw,inner sep=2pt] (char) {#1};}}
\newcommand{\tikzcircle}[2][red,fill=red]{\tikz[baseline=-0.5ex]\draw[#1,radius=#2] (0,0) circle ;}%command for draw text circle coloured
\def\changemargin#1#2{\list{}{\rightmargin#2\leftmargin#1}\item[]}
\let\endchangemargin=\endlist
\makeatletter
\let\originalpart=\part



\newcolumntype{C}[1]{>{\centering\arraybackslash}p{#1}}
\newcolumntype{L}[1]{>{\arraybackslash}p{#1}}
\newcolumntype{R}[1]{>{\raggedleft}p{#1}}
\newcolumntype{G}[1]{>{\centering\arraybackslash\columncolor{gray0}}p{#1}}

\definecolor{gray0}{gray}{0.9}
\definecolor{gray1}{gray}{0.7}
\definecolor{gray2}{gray}{0.4}

\lstset{
	literate = {α}{{$\alpha$}}1 {∆}{{$\Delta$}}1 {θ}{{$\theta$}}1 {η}{{$\eta$}}1 {→}{{$\rightarrow$}}1 {∂}{{$\partial$}}1, %tutti i simboli da usare come codice
	language = Mathematica % linguaggio
}
\hypersetup{
	citebordercolor=red
}


% ----- TITLE
\titleformat*{\section}{\Large\bfseries}
\title{
	\large{University of Trento}\\
	\normalsize{Master in Mechatronics Engineering}\\
	\vspace{0.2cm}
	\large{\textit{Modelling and Simulation of Mechatronics Systems}}\\
	\vspace{0.2cm}
	\Large{\textbf{Development, analysis and optimization of the performance of an innovative driving simulator}}\\
	\vspace{0.25cm}
	\hrule
	\vspace{0.2cm}
	\large{\textbf{Kinematics analysis}}\\	% Title
	\vspace{0.2cm}
	\hrule
}
\author{A. Comoretto \and J. Losi \and S. Valentini}
\date{\vspace{0.5cm}}
% ----- TITLE


\begin{document}
\maketitle

The kinematic analysis...

\section{Kinematics equations}
\label{s:Equations}
In this section the mechanism's behaviour is studied.
In Figure \ref{f:Top-View} is shown the top view of the mechanism.
\begin{figure}[h!]
	\centering
	\includegraphics[width=7cm]{Images/Mechanism_TopView}
	\caption{Top view of the full-mechanism.}
	\label{f:Top-View}
\end{figure}

In first approximation a 2D analysis is conduced, and is refered only to the bodies \circled{A} and \circled{B}.

\subsection{2D-kinematics analysis}
The mechanism studied in the 2D simplification is the marked part in Figure \ref{f:Top-View}, in fact only sub-mechanisms \circled{A} and \circled{B} are taken into account, and the result is shown in Figure \ref{f:2D_Mechanism}.
\begin{figure}[h!]
	\centering
	\includegraphics[width=7cm]{Images/Mechanism_LateralView}
	\caption{2D mechanism.}
	\label{f:2D_Mechanism}
\end{figure}

For the 2D analysis it is chosen to consider the length of the platform as a constant, even if it could be variable, due to its geometry.
More complex analysis are made in the following Section \ref{s:3D-kinematic} during the 3D analysis.

Thanks to this consideration it is possible to say that the full-mechanism is composed by two mirrored sub-mechanisms, made by four sub-bodies, joined by the platform.
So the sub-mechanism studied during the 2D kinematic analysis is shown in Figure \ref{f:Sub-Mechanism}.
\begin{figure*}[h!]
	\centering
	\includegraphics[width=12cm]{Images/Sub-Mechanism}
	\caption{One of the four sub-mechanism of the structure.}
	\label{f:Sub-Mechanism}
\end{figure*}
The same can be done for all the other sub-mechanisms by changing the subscripts.

At a first sight, it is easy to say that the four motors will be the independant variables, they are:
\begin{itemize}
  \item \( s_{A1x}(t) \);
  \item \( s_{A4x}(t) \);
  \item \( s_{B1x}(t) \);
  \item \( s_{B4x}(t) \).
\end{itemize}
Although the simplicity of this consideration it has to be revisited in way to work together with the previous one: in fact consider the length of the platform (named \( L_5\)) constant, means that one of the motor has to become a dependant variable.

In particular \( s_{B1x}(t) \) is considered the dependant variable.

At the same time has to be defined a set of dependant variables.
In particular for the description of the sub-mechanism's behaviour the following elements are chosen:
\begin{itemize}
  \item \( s_{A3x}(t) \) and \( s_{A3y}(t) \) which are the space coordinates of the body 3;
  \item \( \theta_A(t) \) which is the angle of the body 2;
  \item \( B_A(t) \) which is the distance of the body 3 from the upper-left edge of the body 4.
\end{itemize}


\subsection{3D-kinematics analysis}
\label{s:3D-kinematic}

\section{Velocity analysis}

\section{Acceleration analysis}

\section{Effects of the main geometrical parameters}
By varying the position of the actuators it is possible to know which positions can be reached by the end effector, whose set is called \Virgolette{workspace}.
This is one of the main parameters analyzed during the design and modelling of the mechanism: in particular it is required to optimize the workspace and to make it widen to the maximum.

This procedure will lead to determine the geometrical values of the mechanism, which are, from Figure \ref{f:Sub-Mechanism}, \( L_1 \), \( H_1 \), \( L_2 \), \( H_2 \), \( L_4 \) and \( \alpha_4 \).

\subsection{Extremes position problem}
At first, workspace is defined by moving only one actuator, while others are mantained fixed.
This step is repeated for all the three independant variables (\( s_{B1x}(t) \) is dependant, as explained in Section \ref{s:Equations}).

The fixed position of the actuators are defined as follow:
\begin{equation}
  s_{A1x,0} = 0
\end{equation}
\begin{equation}
  s_{A4x,0} = \sqrt{L_2^2-\frac{H_2+H_4-2H_1}{4}} - \frac{L_4}{2}
\end{equation}
and the symmetrical happens for the fixed position for the sub-mechanism \circled{B} (\( s_{A1x,0} = s_{B1x,0} \) and \( s_{A4x,0} = s_{B4x,0} \)).

The first part of this process consists in the definition of the ranges in which the actuators can move whitout generation unassemblable configurations.
The constraints are here summarized:

\begin{equation}
    s_{A1}x(t)+L_1 < s_{A4x}(t)+c_0
\end{equation}
\begin{equation}
    H_4 < s_{A3y}(t)
\end{equation}
\begin{equation}
   s_{A3y}(t) < H_2
\end{equation}
\begin{multline}
    s_{A4x}(t) < s_{A1x}(t)+\\
    +L_2*cos(arctan(\frac{H_2-H_1}{L_2}))
\end{multline}
\begin{equation}
   s_{A4x}(t)+s_{B4x}(t)+2 L_4 < L_e-c_1
\end{equation}
The same is applied for sub-mechanism \circled{B}.
Are also introduced two variables, useful for the optimization part, which are:
\begin{itemize}
  \item \( c_0 \): interpenetration between bodies 1 and 4;
  \item \( c_1 \): minimum distance required between the two sub-mechanisms \circled{A} and \circled{B}.
\end{itemize}

For making this work in an automated way a Maple\textsuperscript{TM} function is written, \texttt{extremes(data,c0,c1)}, which returns a set containing all the space limits for the actuators.







\begin{thebibliography}{}
\bibitem{aVDS}
\Virgolette{\textit{Advanced Vehicle Driving Simulator}}, \textsc{ABDynamics}.

\bibitem{CKAS}
\Virgolette{\textit{6DOF Motion System}}, \textsc{CKAS}.

\bibitem{Kasim}
M. Kasim A. J., \Virgolette{\textit{Design and development of 6-dof motion platform for vehicle driving simulator}}, Universiti Teknologi Malaysia.
\end{thebibliography}
\end{document}
